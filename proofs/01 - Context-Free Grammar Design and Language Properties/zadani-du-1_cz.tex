\documentclass[10pt]{article}
\usepackage[utf8]{inputenc}
\usepackage[T1]{fontenc}
\usepackage{amsmath}
\usepackage{amsfonts}
\usepackage{amssymb}
\usepackage[version=4]{mhchem}
\usepackage{stmaryrd}
\usepackage[margin=0.5in]{geometry}

\begin{document}
\noindent

\noindent
(0,5 bodu) Navrhněte bezkontextovou gramatiku pro následující bezkontextový jazyk:

$$L_1 = \{ w : w \in \{ \mathtt{a, b, c} \}^* \wedge w = w^\mathrm{R} \} \cap \{ u : u \in \{ \mathtt{a,b} \}^* \wedge |u|_{\mathtt{a}} \bmod 3 = 0 \} $$

\noindent
Poznámka: Pomocí zápisu $|x|_a$ značíme počet symbolů $a$ v řetězci $x$.

\begin{description}
   \item 
    \textbf{ŘEŠENÍ:} 
    
        Můžeme si všímnout, že $L_2 := \{ w : w \in \{ \mathtt{a, b, c} \}^* \wedge w = w^\mathrm{R} \}$ obsahuje všechny palindromy, tj:
        $$L_2 = \{\epsilon, a, b, c, aba, bab, cac, abbcbba, ... \}.$$

        $L_3 := \{ u : u \in \{ \mathtt{a,b} \}^* \wedge |u|_{\mathtt{a}} \bmod 3 = 0 \} $ obsahuje všechny řetězce které mají počet symbolů $a$ takový, že nemá žádný zbytek po dělení 3, tj. řetězec musí obsahovat 0, 3, 6, 9 atd. symbolů $a$.
        $$L_3 =  \{\epsilon, b,bb, bbb, bbbb, ..., aaa, abaa, baaabbbbaaba, ...  \}$$

        Průnik $L_2$ a $L_3$ musí tedy obsahovat všechny řetězce $w$, pro které platí, že $w$ je palindrom $\land$ je nad abecedou $\{a, b\}$ ($L_3$ vylučuje všechny řetězce které mají alespoň jeden symbol $c$) $\land$ počet symbolů $a$ má žádný zbytek po dělení 3

        Tj:
        $$L_1 = L_2 \cap L_3 := \{w : w \in \{a, b\}^* \land w = w^R \land |w|_a\ \bmod 3 = 0 \}$$
        Například:
        $$L_1 =\{ b, bb, bbb, ..., aaa, ababa, baaab, aaabaaa, bbaabababaabb, ...\}$$
        
       $ G = (N, \Sigma, P, S ) $ je bezkontextová gramatika, která právě generuje takový jazyk, tj. $L(G) = L_1$, kde:

       
       \begin{align*}
       & N := \{ S, B, C\}  \\
       & \Sigma := \{a, b\} \\
       & P := \{ \\
       & \qquad S \rightarrow \epsilon\ |\ aBa\ |\ bSb\ |\ b, & |w|_a \bmod 3 = 0\\
       & \qquad B \rightarrow a\ |\ bBb\ |\ aCa, & |w|_a \bmod 3 = 2\\
       & \qquad C \rightarrow aa\ |\ bCb\ |\ aSa & |w|_a \bmod 3 = 1\\
       & \}
       \end{align*}
       
       Příklady derivací:

        \begin{align*}
        & S \Rightarrow \epsilon \\
        & S \Rightarrow b \\
        & S \Rightarrow aBa \Rightarrow aaa \\
        & S \Rightarrow aBa \Rightarrow abBba \Rightarrow ababa \\
        & S \Rightarrow aBa \Rightarrow abBba \Rightarrow abbBbba \Rightarrow abbabba \\
        & S \Rightarrow aBa \Rightarrow abBba \Rightarrow abbBbba \Rightarrow abbaCabba \Rightarrow abbaaaabba \\
        & S \Rightarrow aBa \Rightarrow abBba \Rightarrow abbBbba \Rightarrow abbaCabba \Rightarrow abbabCbabba \Rightarrow abbabaababba\\
        & S \Rightarrow aBa \Rightarrow abBba \Rightarrow abbBbba \Rightarrow abbaCabba \Rightarrow abbaaSaabba \Rightarrow abbaabSbaabba \Rightarrow abbaabbbaabba\\
        \end{align*}
        
\end{description}

\noindent\rule{\textwidth}{0.4pt}
\vskip 0.3cm

\noindent
(0,5 bodu) Uvažujte následující dvě tvrzení. U každého tvrzení rozhodněte, zdali je pravdivé, nebo nepravdivé. Následně svou odpověď řádně dokažte. 

\begin{description}
    \item[Tvrzení 1] Každý konečný jazyk je regulární.
    
        \textbf{ŘEŠENÍ:} 

        Pravdivost tohoto tvrzení dokážeme pomocí indukce:

        Nechť máme konečný jazyk $L$, který má $n$ řetězců, $n \in \mathbb{N}_0$.

        $ZK_1:$

        Pro $n = 0$: 
        
        Máme prázdný jazyk, který se dá vygenerovat regulární gramatikou: 
        $$G = ( \{S\}, \{\}, \{\}, S)$$ a tedy je to regulární jazyk.

        Pro $n = 1$:

        Máme jazyk, který obsahuje pouze jeden řetězec. Dokážeme, že takový jazyk je regulární:

        Důkaz provedeme zase indukcí:

        Potřebujeme, pomocí regularní gramatiky vygenerovat jeden řetězec delky $m \in \mathbb{N}_0$, který dle definice je konečná posloupnost symbolů abecedy.

        $ZK_2:$

        Pro $m = 0$, máme prázdný řetězec $\epsilon$, který vygeneruje regulární gramatika:
        $$G = (\{S\},\{\}, \{S\rightarrow \epsilon\},S)$$
        tj. $L(G)=\{\epsilon\}$ je regulární jazyk

        Pro m = 1, máme vygenerovat řetezec $w = w_1$. To umíme pomoci regulární gramatiky:
        $$G = (\{W_1\},\{w_1\}, \{W_1\rightarrow w_1\},W_1)$$

        Pro m = 2, máme vygenerovat řetězec $w = w_{1}w_{2}$. To zase umíme pomoci regulární gramatiky:
        $$G = (\{W_1, W_2\},\{w_1, w_2\}, \{W_1\rightarrow w_{1}W_2, W_2 \rightarrow w_2\},W_1)$$

        Pro m = 3, máme vygenerovat řetězec $w = w_{1}w_{2}w_{3}$. To zase umíme pomoci regulární gramatiky:
        $$G = (\{W_1, W_2, W_3\},\{w_1, w_2, w_3\}, \{W_1\rightarrow w_{1}W_2, W_2 \rightarrow w_{2}W_3, W_3 \rightarrow w_3\},W_1)$$

        $IK_2:$

        Pro m + 1, máme vygenerovat řetězec $w = w_{1}w_{2}w_{3}...w_{m}w_{m+1}$. To uděláme následovně:

        Dle indukčního předpokladu mějme regulární gramatiku G, která generuje jeden řetězec délky $m$ (a tedy i regulární jazyk obsahující tento řetězec):
        \begin{align*}
            & G = ( \\
            &\{W_1, W_2, W_3, W_4, ..., W_m\}, \\
            &\{w_1, w_2, w_3, ... w_m\}, \\
            &\{\\
            &\qquad W_1 \rightarrow w_{1}W_2, \\
            &\qquad W_2 \rightarrow w_{2}W_3, \\
            &\qquad W_3 \rightarrow w_{3}W_4, \\
            &\qquad ... \\
            &\qquad W_{m-1} \rightarrow w_{m-1}W_m, \\
            &\qquad W_{m} \rightarrow w_{m} \\
            &\},\\
            &W_1\\
            &)
        \end{align*}

        Symbol na pozici $m + 1$ vygenerujeme jednoduše tím, že rozšíříme poslední přepisovací pravidlo, a přídáme ještě jedno pravidlo, které vygeneruje symbol $w_{m + 1}$ na posledním místě:

        \begin{align*}
            & G = ( \\
            &\{W_1, W_2, W_3, W_4, ..., W_m\} \cup \{W_{m+1}\}, \\
            &\{w_1, w_2, w_3, ... w_m\} \cup \{w_{m+1}\}, \\
            &\{\\
            &\qquad W_1 \rightarrow w_{1}W_2, \\
            &\qquad W_2 \rightarrow w_{2}W_3, \\
            &\qquad W_3 \rightarrow w_{3}W_4, \\
            &\qquad ... \\
            &\qquad W_{m-1} \rightarrow w_{m-1}W_m, \\
            &\qquad W_{m} \rightarrow w_{m}W_{m+1}, \\
            &\qquad W_{m+1} \rightarrow w_{m+1} \\
            &\},\\
            &W_1\\
            &)
        \end{align*}

        Výsledná gramatika pro $m+1$ symbolů je zase regulární, a tedy i generovaný jazyk je taky regulární.
        
        Tím jsme dokázali, že libovolný jazyk, který má právě jeden řetězec libovolné delky je vždycky regulární.

        $IK_1:$
        
        Dokážeme teď, že jazyk, který má n + 1 řetězců, je taky regulární:
        Víme, že regulární jazyky jsou uzavřené na operaci sjednocení. Ňechť $L_{n+1}$ je jazyk který má $n+1$ řetězců. Potom platí:

        $$L_{n+1} = L_{n} \cup L_{1}$$ kde $L_{n}$ je jazyk který má n řetězců a $L_{1}$ je jazyk který má 1 řetězec.
        
        Z indkučního předpokladu platí, že $L_{n}$ je regulární jazyk. Ze zakladního kroku $ZK_1$ víme, že i $L_1$ je regulární. Potom i sjednocení $L_n$ a $L_1$ bude jazyk regulární, tj. jazyk $L_{n+1}$ je regulární.

        Tím jsme dokázali, že pro každý konečný jazyk platí, že je regulární.
        
        
    \item[Tvrzení 2] Každý regulární jazyk je konečný.    
    
        \textbf{ŘEŠENÍ:} 

            Dokážeme sporem, že tvrzení neplatí:

            Stačí, abychom našli regulární gramatiku, která bude generovat nekonečně mnoho řetězců:
            $$G = ( \{S\}, \{a\}, \{ S \rightarrow a\ |\ aS\  \}, S ) $$

            Potom:
            $$L(G) = \{a, aa, aaa, aaaa, aaaaa, aaaaaa, ...\}$$ obsahuje nekonečně mnoho řetězců obsahujících pouze symboly $a$, a tedy tento regulární jazyk generovaný regulární gramatikou není konečný, což je ve sporu s tvrzením, že každý regulární jazyk je konečný.
    
\end{description}




\end{document}